% Options for packages loaded elsewhere
\PassOptionsToPackage{unicode}{hyperref}
\PassOptionsToPackage{hyphens}{url}
%
\documentclass[
]{book}
\usepackage{lmodern}
\usepackage{amssymb,amsmath}
\usepackage{ifxetex,ifluatex}
\ifnum 0\ifxetex 1\fi\ifluatex 1\fi=0 % if pdftex
  \usepackage[T1]{fontenc}
  \usepackage[utf8]{inputenc}
  \usepackage{textcomp} % provide euro and other symbols
\else % if luatex or xetex
  \usepackage{unicode-math}
  \defaultfontfeatures{Scale=MatchLowercase}
  \defaultfontfeatures[\rmfamily]{Ligatures=TeX,Scale=1}
\fi
% Use upquote if available, for straight quotes in verbatim environments
\IfFileExists{upquote.sty}{\usepackage{upquote}}{}
\IfFileExists{microtype.sty}{% use microtype if available
  \usepackage[]{microtype}
  \UseMicrotypeSet[protrusion]{basicmath} % disable protrusion for tt fonts
}{}
\makeatletter
\@ifundefined{KOMAClassName}{% if non-KOMA class
  \IfFileExists{parskip.sty}{%
    \usepackage{parskip}
  }{% else
    \setlength{\parindent}{0pt}
    \setlength{\parskip}{6pt plus 2pt minus 1pt}}
}{% if KOMA class
  \KOMAoptions{parskip=half}}
\makeatother
\usepackage{xcolor}
\IfFileExists{xurl.sty}{\usepackage{xurl}}{} % add URL line breaks if available
\IfFileExists{bookmark.sty}{\usepackage{bookmark}}{\usepackage{hyperref}}
\hypersetup{
  pdftitle={A Serious Call to a Devout and Holy Life},
  pdfauthor={William Law, A.M.},
  hidelinks,
  pdfcreator={LaTeX via pandoc}}
\urlstyle{same} % disable monospaced font for URLs
\usepackage{longtable,booktabs}
% Correct order of tables after \paragraph or \subparagraph
\usepackage{etoolbox}
\makeatletter
\patchcmd\longtable{\par}{\if@noskipsec\mbox{}\fi\par}{}{}
\makeatother
% Allow footnotes in longtable head/foot
\IfFileExists{footnotehyper.sty}{\usepackage{footnotehyper}}{\usepackage{footnote}}
\makesavenoteenv{longtable}
\usepackage{graphicx}
\makeatletter
\def\maxwidth{\ifdim\Gin@nat@width>\linewidth\linewidth\else\Gin@nat@width\fi}
\def\maxheight{\ifdim\Gin@nat@height>\textheight\textheight\else\Gin@nat@height\fi}
\makeatother
% Scale images if necessary, so that they will not overflow the page
% margins by default, and it is still possible to overwrite the defaults
% using explicit options in \includegraphics[width, height, ...]{}
\setkeys{Gin}{width=\maxwidth,height=\maxheight,keepaspectratio}
% Set default figure placement to htbp
\makeatletter
\def\fps@figure{htbp}
\makeatother
\setlength{\emergencystretch}{3em} % prevent overfull lines
\providecommand{\tightlist}{%
  \setlength{\itemsep}{0pt}\setlength{\parskip}{0pt}}
\setcounter{secnumdepth}{5}
% DEFINE PHYSICAL DOCUMENT SETTINGS HD
% media settings
\usepackage[paperwidth=5.5in, paperheight=8.5in]{geometry}

\usepackage{booktabs}
\usepackage{amsthm}
\makeatletter
\def\thm@space@setup{%
  \thm@preskip=8pt plus 2pt minus 4pt
  \thm@postskip=\thm@preskip
}

\usepackage{titling}
\usepackage{pdfpages}
\IfFileExists{./cover.pdf}{
  \newcommand{\myCover}{./cover.pdf}}
  {\IfFileExists{./cover.jpg}{
    \newcommand{\myCover}{./cover.jpg}}
    {\IfFileExists{./cover.png}{
      \newcommand{\myCover}{./cover.png}}{}
    }
  }
\@ifundefined{myCover}
{}
{
\pretitle{\begin{center}\includepdf{\myCover}}
\posttitle{\end{center}\setcounter{page}{0}}
\usepackage{atbegshi}% http://ctan.org/pkg/atbegshi
\AtBeginDocument{\AtBeginShipoutNext{\AtBeginShipoutDiscard}}
}
\clearpage\pagenumbering{roman}

\newenvironment{poetry}[0]{\par\leftskip=2em\rightskip=2em}{\par\medskip}

\setmainfont{Calluna}
\newfontfamily\greekfont[Script=Greek]{LiberationSerif}

\makeatother

\frontmatter
\ifluatex
  \usepackage{selnolig}  % disable illegal ligatures
\fi
\usepackage[]{natbib}
\bibliographystyle{plainnat}

\title{A Serious Call to a Devout and Holy Life}
\author{William Law, A.M.}
\date{1821}

\begin{document}
\maketitle

\mainmatter
\pagenumbering{roman}

{
\setcounter{tocdepth}{1}
\tableofcontents
}
\hypertarget{about-this-book}{%
\chapter*{About this book}\label{about-this-book}}
\addcontentsline{toc}{chapter}{About this book}

Originally published as ``This Is An Empty Book'' in \emph{Warhorn Classics Anthology} (Bayly, McNeilly, Weeks, et al, 2020), 219--283.

Republished by \href{https://classics.warhornmedia.com/}{Warhorn Classics}---making classic Christian content available online for \textsc{free} in high quality, readable formats.

The latest version of this book can always be found \href{https://warhornmedia.github.io/law-serious-call/}{here} in many electronic formats for your reading convenience on any device.

\hypertarget{downloads}{%
\subsubsection*{Downloads}\label{downloads}}
\addcontentsline{toc}{subsubsection}{Downloads}

\href{https://warhornmedia.github.io/law-serious-call//Law-A_Serious_Call_To_A_Devout_And_Holy_Life.pdf}{Download PDF}

\href{https://warhornmedia.github.io/law-serious-call//Law-A_Serious_Call_To_A_Devout_And_Holy_Life.epub}{Download ePub}

\hypertarget{original}{%
\subsubsection*{Original}\label{original}}
\addcontentsline{toc}{subsubsection}{Original}

Scanned images of the original printing of this book (our authoritative source) are available \href{https://archive.org/details/aseriouscalltoa00unkngoog}{here}.

\hypertarget{text-status}{%
\subsubsection*{Text Status}\label{text-status}}
\addcontentsline{toc}{subsubsection}{Text Status}

\begin{enumerate}
\def\labelenumi{\arabic{enumi}.}
\setcounter{enumi}{-1}
\tightlist
\item
  \textbf{Current status:} --\textgreater{} Unstarted - Empty project files created, but that's it.
\item
  Rough - Metadata entered, computer-generated text from scans entered. Unstructured, and likely to include many errors.
\item
  Cleaned - Text has been structured, footnotes and endnotes linked, and extra line-breaks, hyphens, page numbers and other artifacts removed.
\item
  Complete - Carefully proofed, including any foreign language quotes, footnotes, and endnotes.
\end{enumerate}

\hypertarget{editorial-notes}{%
\subsubsection*{Editorial Notes}\label{editorial-notes}}
\addcontentsline{toc}{subsubsection}{Editorial Notes}

\begin{enumerate}
\def\labelenumi{\arabic{enumi}.}
\tightlist
\item
  Headings: Structuring this book for easier browsing required adding titles to various sections. Any heading text in {[}brackets{]} was added in the editing process.
\item
  If you're reading this, you can't trust the editorial notes. ;)
\end{enumerate}

For more information on our editorial commitments and process, please \href{https://classics.warhornmedia.com/editorial}{click here}.

\hypertarget{support-warhorn-classics}{%
\subsubsection*{Support Warhorn Classics}\label{support-warhorn-classics}}
\addcontentsline{toc}{subsubsection}{Support Warhorn Classics}

We hope this book is a blessing to you. If it is, please \href{https://warhornmedia.com/give}{make a one-time or recurring contribution} right now, sponsor a book from our upcoming list, or volunteer your proofreading or technical skills to help produce more content. Contact \href{mailto:lucas@beggarsborn.com}{Lucas Weeks} to get involved.

\clearpage
\setcounter{page}{1}\pagenumbering{arabic}

\hypertarget{chapter-1}{%
\chapter{Chapter 1}\label{chapter-1}}

Concerning the Nature and Extent of Christian Devotio

Devotion is neither private nor public prayer; but prayers, whether pmate or public, are particular parta or instances of devotion. Devotion ligitifies a life giren or devoted to God.

He therefore is the devout man, who lives do longei to bis own will, or the wiiy and apirit of the world, but to the sole will of God, who considers God in everv thing, who serves God in every thing, who nutkes all , the parts of his common life, parts of piety, by doing erery thing in the name of Got!, luid under such rules » ire conformable to his glory.

We readily acknowledge, that God alone is to be the mie and measure of our prayers, that in them we are to loot wholly unto him, and act wholly for him, that we are only to pray in such a manner, for such things, and iuch ends as are suitable to his glory.

Now let any one but iind out the reason why he is to be thus strictly pious in his prayers, and he will find the same as strong a reason to be as strictly pious in all the other paifs of bis life. For there is not the least shadow of a reason, why we should make God the rule and measure otjijir prayers, why we should then look wholly `\^{}■\textbar B,bim?Smd pray according' to his will ; but what

2« A 9ER10U3 CALL TO A

equallj proves it necessary for us to look wholly nnW God, and make him the rule and measure of \^{}1 the »ther actions of our life. For any ways of life, any em- ployment of our talents, whether of our parts, onr time or money, that is not strictly according to the will of God, that is not for such ends as are suitable to his glory, are as great absurdities and failings, as prayers that are not according to the will of God. For there is qo other reason, why our prayers should be according to the will of God, why they should have nothing in them, but what is wise, and ho!y, and heavenly, there is no other reason for this, but that our lives may be of the same na- ture, full of the same wisdom, holiness and heavenly tempers, that we may live unto God in the same' spinl that we pray unto him. Were it not our strict duty to Uve by reason, to devote all the actions of our lives to God, were it not absolutely necessary to walk before him ID wisdom and holiness and all heavenly conversatioD, doing every thing in his name, and for his gloiy, there would be no excellency or wisdom in the most heavenly prayers. Nay, such prayers would be absurdities, they would be like prayers for wings, when it was no part of our duty to Bj.

As sure therefore as there is any wisdom in praying for the spirit of God, so sure is it, that we are to make that Spirit the rule of all onr actions ; as snre as it is our duty to look wholly unto God in our prayers, so sure is it, that it is our duty to live wholly unto God in our Sves. But we can no more he said to live unto God, unless we live unto him in aifthe ordinary actions of our life, unless b\^{} be the rule and measure of all our ways, than we cau be said lo pray unto God, unless our prayers look wholly unto him. So that unreasonable and ab- ■urd ways of Ufe, whether in labour or diversion, wheth- er they consume our time or our money, are like unrea- sonable and absurd prayers, and are as truly an offence onto God.

It is for want of knowing,\textasciitilde or at least considerii\^{} this, that we see such a mixture of ridicule in the lives ofmany people. You see them strict as to some times and places of de'Votion ; but when the service of the church is over, they are but Uke those that seldom or sever cgo»

DEVtSOT AStD HOLT UFE. tt

itiere. la their way of life, their maDoer of ipendiop their time aad taoaej, ia their cares and fean, iQ their pleasures and initulgeaces, io their labour aDd diTenions tbey are tike therest of the world. Tliis makea the loose ' part of the world generally make a jest of those that are devout, becatise they see their derotioa goes do ferther tbaa their prayers, and that when they are overy they U?e no more unto God, till the time of prayer retoms again ; but Uve by the same hgmonr and fancy, and in as ManeDjoyiBentofaU the ibllies of Ufc, as other people. This is the reason why they are the jest and scorn of careless and worldly people ; not because they are really ' derot\^{}d to God, but because they appear to hare no otb- ei devotion, but thatofoccasiwial prayen.

Julius is very fearful of missing prayers ; all the parish Buppodes Julius to be sick, if he is not at church. But if yoa was to ask him why he spends the rest of his time hy boinour or chance 1 why he is. a companion of the ailUest people in their most ully pleasures ? why is he ready for every impertinent entertainment and diverrion 1 If you was to ask him why there is no amusement too trifluig to please him ? why he is busy at alf Stalls and assemblies I why he gives himself up to an idle gossiping ctmveraation 1 why he Uves in foolish friendships and fondness for particular persons, that neither want nor deserve any particular kindness? why he allows him- self in foolish hatreds and resentment) against par1ici\textgreater{} lu persons, without considering that he is to love every body as himself i If you ask him why he never puts hu coaversation, his time, and fortune under the rules (tf religion, JuUus has no more to say for himself, than the most disorderly person. For the whole tenor of Scrip- ture lies as directly against such a life, as .against de- bauchery and intemperance : He that hves in such a course of idleness and folly, lives no more according to the religion of Jesus Christ, than he that lives in gluttony and intemperance.

If a man was to tell Julius" that there was no occasion for so much constancy at prayers, and that he might, without any harm to himself, neglect the service of the church, as the generabty of people do, Julius wonld think such a one to be no Cbristiitn, aiul that be ougb* 3* ■

30 i. SERIOUS CALL TO k

to aroid his compwy. Bat if a person only tetla him . that he may live as the generality of the worid doea. that se may eajoy himself as others do ; - that he may spend his time and mooey as people of fashion do, that he may conforna to the follies and frailties of the generality, and gratify his tempera and passions aa most people do, Julius nevev suspece that man to want a Chmtian B{[}Hrit, Or that be is doing the devil's work. ,

And yet if Jalius was (o read all the Nevf Testament from the beginning to the end, he would fkid his coime of life condemned in every page of it.

And indeed there cannot a\^{}y thing be imagined more absurd in itself, than wise and sublime, and heavenly\^{}

f)rayers added to a life of ranify and folly, where neither abour nor diversions, neither time nor money, are under the direction of the wisdom and heavenly tempers of oar prayers. If we were to see a man pretending to act wholly with regard .to Ood in every thing that he did, that would neither spend time nor money, or take any labour or diversion, but ao far as he could act' according to strict principles of reason and piety, andyetat-tfae same time neglect all prayer, whether public or private, should we not be amazed at such a man, and wonder bow ha could have so much folly along with so much religion T

Yet this is as reasonable as for any person to pretend 4o strictness in devotion, to be careful of observing times and places of prayer, and yet letting the rest of his life, fail time and labour, bie talents and money be (Ksposed of, without any regard to strict rules of piety and devo- tion, for it is as great an absurdity to soppose holy pray- ers, and divine petitions, without an holiness of life suita- ble to them, as to auppose ao holy and divine life without prayers.

Let any one therefore think, how easily he could coo- tate a mas that pretended a great strictness of life with- •ut prayer, and the same arguments will as plainly con- fute another, that preteni\^{} to strictness of prayer, without canying the same strictness into every othec ' part of life. For to be weak and foolish in spen\^{}ng our time and fortwie is no greater a mistake, than to be weak a»d fonksk » r«lati«a to 9»r prayers. And to

ItEVOtri! AND HOLT UFE. 3t

riloir onraelves in any wajrs of life that neither are nor can lie offered to Qpi, is the same irreligion as to neglect oar prayers, or use them in soch a manner, as make\# tbem aa offering unworthy of God.

The thort of the nmtte'r it this, either reason and re- Kgioa prescritie rules and ends to all the ordinary actiou of oar hfe, or they do oot : If they do, then it is as De- cenary to govera all our actions hy those rales, as it is necessary to worship God. ' For if religion teaches as any tiuog coocerniDg eating and drinking, or spending OVF tiine and money, if it teaches us how we are to use and contemn the world ; if it tells us what tempers we are to have in common life, how we are to be disposed towards all people. How we are to behave towards the Kick, tbe poor, the old and destitute ; if it tells us whom ne are to treat with a particular love, whom we are to regard with a particular esteem: if it tells ns how wft are to treat our enemies, and how we are to mortify and deny ourselves, he may be very weak, that can think the\^{}e parts of religion are not to be observed with aa Bwch exactness, as any doctrine that relates to prayers.

It is very observable, that there is not one command ID all the gospel for public worship ; and perhaps it is a ■ duly that is least insisted upon in scripture of any other. The frequent attendance at it is never so much as men- tioned in all the New Testament. Whereas that religion Or devotion, which is to govern the ordinary actions of our life, is to be found in almost every verse of scripture, Oor blessed Saviour and his apostles are wholly taken op in doctrines that relate to common life. They call OS to renounce the world, and differ in every temper and way of life, from the spirit and way of the world. To renounce all its goods, to fear none of its evils, to reject its joys, and have no value for its happiness. ' To be as new bom babes, that are bom into a new state of things, to live as pilgrims in spiritual watching, in holy fenr, and heavenly aspiring after another life. To take up our daily cross, to deny ourselves, to profess the blessednes* of mourning, to seek tbe blessedness of poverty of spirit. To forsake the pride and vanity of riches, to take no thought for tbe morrow, to live in the profoundeet state of humility, to rejoice in worldly sufferings. To reject

aa A SERIOUS call to a

the lust of the flesh, the lust of the eyes, and the pride of life;* to bear injuries, to forgive aod bless ourenemiea,'\,' 'And to love mankind as God loveth them. Ta giv.e up our whole hearts and afiectioos to G«d, aod strive to eater through the straight gate into a life of eternal glory.

This is the commoD devotion which our blessed Sa- viour taught, in order to make it t)ie common life of all Christians. Is it not therefore exceeding strange, that people should place so much piety in tlie attendance of puhUc worship, concerning which there is not one pre- cept of our Lord's to he found, and yet neglect these- common duties of our ordinary life, which are command- ed in every page of. the gospel ? .1 call these duties the devotion of our common life, because if they are to be practised, they must be made parts of our common life, they can have no place any where else.

If contempt of the world, and heavenly affection, is a necessary temper of Christians, it is necessary that this temper appear in the nhole course of their lives, in their manner of usii\^{} the world, because it can have no place any where else.

If self-denial be a condition of salvation, j\textsuperscript{l}t- that would be saved must make it a part of theif Ofdinfiry hfe. If humility be a Christian duty, then, the.cOmmoa life of a Christian is to be .a constant course of humility in all its kiqds. If poverty of spirit be necessary, it must be the spirit and temper of every day' of our bves. If we are to relieve the naked, the sick, and the prisoner, it must be ihc common charity of our lives, as far as we can render ourselves able to perform it. If we are to love our enemies, we must make our common Ufe a visible ex- ercise and demonstration of that love. If content and thankfulness,'if the patient bearing of evil be duties to God, they are the duties of every day, and in every cir- cumstance of our life. If we are to be wise and holy as the new-bom sons of God, we can no otherwise be so\^{} but by renouncing every thing that is foolish and vain ta every partof our common life. If we are to be in Christ new crealure?, we must shew that we are so, by having new ways of living in the wori(l\^{} If we are to follow

mGooglc

DEVOUTf AND HOLY LtFE . " 3g

Christ, it must be in our commoa way of spending; ever;

Thus it is in all the virtaes and holy tempen of Chrit- tiaiuQr, they are not ours, uoless they be the virtues and lempeis of Our ordinary life. So that Christiani\^{} is ho far nam leaving as to live in the common ways of life, conforming to the folly of customs, and gratifying the passions and tempeTs which the spirit of the world de- lights in, it 19 so far from indulging ns in any of these diiQji, that all its virtues which it makes necessary to jalration, are only so m-.iny ways of living above, and contrary to the worlil in all the common actions of our fife. If our common hfe is not a common course of hu- mility, self-denial, renunciation of the world, poverty of Bpirit,' and heavenly affection, we do not live tiie lives of Christians.

But yet though it is thus plain, that this and this alone is Chmtiiinity, an uniform, open, and visible practice of sJ! these virtaes ; yet it is as plain, that there is little or nothing of this to be found, even amongst the better sort of people. You 'see them often at church, and pleased with line preachers ; but look into their lives, and'yon see them iust'the same sort of people as others are, that make no pretences to devotion. The difierence that yo» find betwixt theoi. Is only the difierence of their natural tempers. They have the same taste of the world, the same worldly cares, and fears, and joys; they have the same turn of mind, equally vain in their desires. You see the same fondness for state and equipage, the same pride and vanity of dress, the same self-love and indul- gence, the same foolish friendships and groundless ha- treds, the same levity of mind and trifling spirit, the same fondness of diversions, the same idle dispositions and vain ways of spending their tirite in visiting and conver- sation, as the rest of 3ie world, that make no pretences to devotion.

1 do not mean tfafs comparison betwixt people seem- ingly good and professed rakes, but betwixt people of sober Wtes. Let us take an instance in two modest wo- men: let it he supposed, that one of them is careful of times* of devotion, and observes them through a sense of duty, and that the other has «o hearij concera about ity

Si A SERIOUS CALL T* A

fent is at church Mldetn or (Aea, juat tu it happein\^{}-f Now it is a very easy thing to see this difierence betwixt -'these persons. But when you have seen this\^{} can yon find any farther ditference betwixt them? Can you find that their common life is of a dtfiereut Jdndl Are not the tempers, and customa, and manners of the oiie, of the same kind as of the other? Do they live as if they belonged to different worlds, had different views in that beads, and different rules and measures of all their ac- tions ? Have they not the same goods and evils, are they not pleased and displeased in the same manner, and for the same things? Do they not Uve in the same course of life 1 Daeg one seem to be of this world, looking at the things that are temporal, and the other to be of another world, looking wholly at the things that are eternal ?-\textasciitilde{} Does the one live in pleasure, delighting herself in show or dress, and the other live in selt-denial and mortifica- tion, renouncing every thing that lo\textless diB like vanity either of person, dress, or carriage ? Does the one follow pub- lic diversions, and trifle away her time in idle visits and corrupt conversation ; and does the trther slndy all the arts of improving her time, living in prayer and watch* ing, and such good works as may make all her time turn to her advantage, and be placed to her account at the last day? Is the one careless of expense, and glad to be able

to adorn herself with every costly ornament of* dress 7

and does the other consider her fortune as a talent given her by God, Which is to be improved religiflusiy, and no more to be spent in vain and needless omameots, than it is to be buried in the earth t

Where must you look to find one person of religion differing in this manner, from another that has none ? --- And yet, if they do not differ in these things, which are here related, can it with any sense be said, the one is a good Christian and the other not t

Take another instance amongst the men. Leo has a great deal of good nature, has kept what they call good company, hates every thing that is false and base ; Is ve- ry generous and brave to his friends, but has concerned himself so little with religion, that heiiardly knows the difference betwizt a Jew and a Christian.

Eusebius, on the other hand, has had early imprcsaioDs

EBVOOT AStt HOLT LITE \%

af pe\^{}itin, and bays bodu of deToUoa. lie eUi *al\textbar{} i «f all the teaata and fasts of the church, aad knom die nanesofmost men that hava been eminent for pie\^{}.--- Too aerer bear him fwear or make a Joote jest ; and when he t\^{}Ju of rehg\^{}on, he talkftof it, m ofa matter of flke last coDcein.

Here yoa see that one peraon has religion enoogb, according to the w\^{} of the World, to be recktmed a pt oiH Christian, wd the other is so far from all appearaacfl pf religion, that he may fairly be reckoned a Heathen) and yet if yoa look into dieir common hfe, if you esaa\^{} it\textgreater e their chief and roling ten\^{}\textgreater eFs in the greatest articles ■ oflife, or the ereatwl doctrines of Christianily, you mil find the least ditn:{[}«n(-« imaginable.

Consider them with regard to the use t\^{} the world, be- caose there is what every body can see.

Now to have right notions add tempera with relation to this worid, is as eiaenlial to rcLigioo, as ta have right uotioDS of God. And it is as possible for a man to wor- diip a crocodile, and yet be a pious man, as to have his auctions set upiM this world, and yet be a good Chris- tiao.

But now, if you consider Leo and Eusebius in this rd- ■pect, you nill find .them exactly alike, seeking, using, sttd enjoyii\^{}; all that can be got in this world, in the ■ame manner and for the same ends. You will find that riches, prosperity, pleasures, indulgences, state, equip- \^{}e, and faouMir are just as much the happiness of Euse- Woe as they Ce of Leo. And yet if Christianity has not changed a man's mind and temper with relation to these dungs, what can we say that it has done for him ?

For if the doctrines of Christianity were practised, Oiey wonld Hiake a nan as different from other people as to all worldly tempers, sensual pleasures, and the pride of life, as a wise man is different from s natural ; 'it would be as easy a thing to know a Christian by his out- ward course of life, as it is now difficult to find any body that Ures it For it is notorious that Christians are now not only tike other men in their frailties and infirmi- ties, this might be in some degree excusable ; but the complaint is, they are like heathens in all the main and Gtyef articleg of \textbar beir lires. They enjoy the irof Id, w4

36 A SERIOUS CALL TO A

Jire every day in lie same tempere, mm) the same designs, `Bod the same indulgencei, as 'they did who knew not God, nor of any happiness in aoother life. Every body, that is capable of any reSection, mast hftTe observed, that this is generally \textless the state even of devout people, whether men or wom«i. You may see them different ftoiD other people so far as to times and places o( prayer, but generally like the rest of the world in all the other parts of their lives. That is, adding Chri\^{}an devotion to an heathen hfe : 1 have the authority of our blessed Saviour for this remark, where he says, Take no thought, toying what shalt we «U, or 'what thall we' drink., or wher\^{} TDitkUthaiiwebe clothed? for after all these things do the Gtntitee *eek. But if to be thus aflecMd even with the neceBsary things of this \textbar ife, shews that we are not yet of a Christian spirit, bat are like tb« heathens ; surely to enjoy the vani\^{} and fblly of the world aslbey did, to be like them in the main chief tempers of our lives,- in selA love end indu\^{}nce, in sensual pleasures and diversions, in (he vani\^{} of dress, (he love of show and greatness, or any o(her gaudy distinction of fortune, is a much greater sign of a heathen temper. And consequently they who ' add devotion to such a life, must be said to pray as Christians, but Uve as heatbeus.

\end{document}
